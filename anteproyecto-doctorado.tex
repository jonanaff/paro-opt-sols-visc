\documentclass{article}
\usepackage{graphicx} % Required for inserting images
\usepackage[usenames]{color}
\usepackage{graphicx} 
\usepackage{amsfonts}
\usepackage{amsmath}
\usepackage{mathabx}
\usepackage{mathtools}
\usepackage{enumerate}
\usepackage{graphicx}
\usepackage{enumerate}
\usepackage{hyperref}
\usepackage[spanish]{babel}
\usepackage{amsthm}
\renewcommand*{\proofname}{Demostración}
\newtheorem{theorem}{Theorem}[section]
\newtheorem{teorema}[theorem]{Teorema}
\newtheorem{corolario}[theorem]{Corolario}
\newtheorem{lema}[theorem]{Lema}
\newtheorem{prop}[theorem]{Proposición}
\newtheorem{obs}[theorem]{Observación}
\numberwithin{equation}{section}
\newtheorem{defi}{Definición}[section]
\usepackage[spanish]{babel}

\title{Paro óptimo y soluciones de viscosidad}
\author{Jonathan Naffrichoux}
\date{Mayo 2024}

\begin{document}
\maketitle 

\section{Descripción del proyecto}
La conexión entre problemas de paro óptimo y ecuaciones diferenciales parciales con frontera libre se ha explorado extensivamente en las últimas décadas. Un ejemplo remarcable es el de \cite{peskir2006optimal}, donde este tipo de problemas se traducen a resolver ecuaciones integro-diferenciales, con la sutiliza de que las condiciones de frontera dependen de una curva que cambia con el tiempo. Entonces se puede plantear la siguiente pregunta: ¿qué ocurre si el dominio en cuestión es difícil de describir? Parametrizar superficies en dimensiones bajas puede ser una tarea sencilla, ¿pero qué sucede si el problema de paro óptimo se plantea en dimensiones altas? Estas problemáticas motivan el presente proyecto de investigación.

Las condiciones de frontera libre surgen por una razón fundamental, principalmente, de la estrategia propuesta en \cite{peskir2006optimal} para maximizar la función de valor. En este trabajo proponemos codificar estas condiciones de frontera por medio de un término no lineal que se incorporará a una EDP parabólica. 
En la actualidad, existen métodos muy eficientes para resolver este tipo de ecuaciones, como lo son las \emph{soluciones de viscosidad} introducidas por  Pierre-Louis Lions y Michael G. Crandall al inicio de los 1980s. Esta noción permite probar existencia y unicidad de ecuaciones que en un sentido tradicional estarían mal planteadas (aceptan una infinidad de soluciones Lipschitz continuas). La herramienta fundamental que se utiliza en este contexto son los principios del máximo, que sí se pueden extender más allá del laplaciano a operadores completamente no lineales.

De hecho, ya se han estudiado ligas entre ecuaciones parabólicas totalmente no lineales y problemas de \emph{control óptimo}, como lo hace S. Peng en \cite{peng2010nonlinear}. Su enfoque consiste en determinar estas soluciones como supremos de funcionales lineales, lo cual da lugar al novedoso concepto de \emph{esperanza no lineal}. Sin embargo, la conexión con problemas de \emph{paro óptimo} parece no haberse establecido hasta el momento.
Otro problema abierto que se puede plantear dentro de este proyecto de investigación es si existe una conexión parecida cuando reemplazamos el laplaciano clásico por un laplaciano fraccionario. En este caso se esperaría que la función de valor fuera un procesos de Lévy no lineal.

\section{Contexto científico}


Siguiendo la notación de \cite{peskir2006optimal}, sea $(X_n)_{n\in\mathbb{N}}$ una cadena de Markov definida sobre un espacio filtrado $(\Omega,\mathcal{F},(\mathcal{F}_n)_{n\in \mathbb{N}},\mathbb{P}_x)$ y que toma valores en un conjunto numerable $E$.

Sea $G : E \to \mathbb{R}$ una función medible que satisface 
\begin{align}\label{G-esperanza}
    \mathbb{E}_x\left[\sup_{0\leq n\leq N}|G(X_n)|\right] < \infty, \quad \forall x \in \mathbb{E},\: \forall N \in \mathbb{N}.
\end{align}
$G$ se conoce como la \emph{función de ganancia}. El \emph{problema de paro óptimo} (en horizonte finito) que se estudia en \cite{peskir2006optimal} es
\begin{align}\label{paro-optimo}
    V^N(x) = \sup_{0\leq \tau \leq N}\mathbb{E}_x\left[G(X_\tau)\right],
\end{align}
donde el supremo se toma sobre todos los tiempos de paro $\tau$ definidos con respecto a la filtración $(\mathcal{F}_n)_{n\in\mathbb{N}}$ y que toman valores entre $0$ y $N$. Resolver este probema involucra calcular el valor de $V^N(x)$ así como encontrar el tiempo de paro óptimo. La función $V^N(x)$ se conoce como la \emph{función de valor}. Denotando por $TF(x) = \mathbb{E}_x\left[F(X_1)\right]$ al operador de transición de $(X_n)$, la célebre recursión de Wald-Bellman proporciona un algoritmo para calcular la función de valor por medio de
\begin{align}\label{wb}
    V^n(x) = \max\{G(x), TV^{n-1}(x)\}, \quad x \in E.
\end{align}

Los autores de \cite{peskir2006optimal} relacionan el problema \eqref{paro-optimo} con ecuaciones integro-diferenciales que satisfacen condiciones de frontera libre. Para reinterpretar el problema en términos de ecuaciones parabólicas totalmente no lineales, optaremos por la siguiente reinterpretación de la fórmula de Wald-Bellman
\begin{align}\label{nueva-recurrencia}
        V^n(x) = \max\{V^{n-1}(x), TV^{n-1}(x)\}, \quad x \in E, \: 0\leq n\leq N.
    \end{align}
Introduciendo la notación $\partial_nV^n(x) := V^{n+1}(x)-V^n(x)$, podemos reescribirla como
\begin{align}\label{no-lineal}
    \partial_nV^n(x) = \max\{0, TV^{n}(x)-V^{n}(x)\}.
\end{align}
En el contexto de \cite{lawler2010random}, el operador
\begin{align*}
   \mathcal{L}F(x) := TF(x)-F(x),
\end{align*}
es conocido como el \emph{laplaciano discreto}. Así que la recursión de Wald-Bellman se puede entender como
\begin{align}\label{discreto}
    \partial_nV^n(x) = \max\{0, \mathcal{L}V^{n}(x)\}.
\end{align}

Esta expresión sugiere que para resolver el problema de paro 
óptimo en el caso continuo, se puede resolver la ecuación
\begin{align}\label{parabolic-non-linear}
\left\{
\begin{array}{cc}
    \partial_t V = \max\{0,\Delta V\}, & \text{en} \quad  \mathbb{R}^+\times \Omega, \\
    V(0,x) = G(x) & \text{en} \quad \Omega,
\end{array}
\right.
\end{align}
donde $\Omega$ denota un subconjunto abierto de $\mathbb{R}^N$.
De esta manera evitamos lidiar con condiciones de frontera libre y nos enfocamos exclusivamente en resolver EDPs parabólicas totalmente no lineales. 

Para probar que la ecuación \eqref{parabolic-non-linear} está bien planteada requeriremos del concepto de  solución de viscosidad, el cual motivaremos a continuación. Supongamos por el momento que $V$ es lo suficientemente regular y consideremos la función 
\begin{align*}
F : \mathbb{S}(N) \to \mathbb{R}, \quad F(X) = \max\{0,Tr(X)\}, 
\end{align*}
donde $\mathbb{S}(N)$ denota el espacio de matrices simétricas de $N\times N$ con entradas reales y $Tr$ denota el operador traza.  Entonces podemos reescribir la ecuación \eqref{parabolic-non-linear} como $V_t = F(D^2V)$, donde $D^2V$ es la matriz hessiana de $V$. Es bien sabido que el operador traza es no decreciente, así que $F$ es no decreciente (donde una matriz se considera no negativa si es semi positiva definida). 

Sea $\psi \in C^{1,2}((0,T)\times\Omega)$ y supongamos que $V-\psi$ tiene un máximo local en $(t_0,x_0) \in (0,T)\times \Omega$. Entonces el principio del máximo clásico asegura que
\begin{align*}
    \partial_t(V-\psi)(t_0,x_0) &= 0 \quad \text{en } \mathbb{R}^{N+1}, \\
    D(V-\psi)(t_0,x_0) &= 0 \quad \text{en }\mathbb{R}^{N+1}, \\\
    D^2(V-\psi)(t_0,x_0) &\leq 0 \quad \text{en } \mathbb{S}(N).
\end{align*}
Suponiendo de antemano que $V$ es una solución de \eqref{parabolic-non-linear}, podemos hacer la siguiente observación
\begin{align*}
    0 = \partial_tV - F(D^2V) \geq \partial_t\psi - F(D^2\psi).
\end{align*}
Note que la desigualdad 
\begin{align}\label{desigualdad}
0 \geq \partial_t\psi - F(D^2\psi)
\end{align}
es independiente de $V$ y tiene sentido aún cuando $V$ no sea diferenciable. Puesto de otra manera, se han transferido las condiciones de regularidad a la función de prueba $\psi$, lo cual da una idea de cómo se debería definir la solución de una ecuación totalmente no lineal.  Si la desigualdad \eqref{desigualdad} se cumple para una función $\psi$ arbitraria que cumple las propiedades antes descritas, y además se satisface $V(0,x) \leq G(x)$ para toda $x \in \Omega$, diremos que $V$ es una \emph{subsolución de viscosidad} de \eqref{parabolic-non-linear}. Análogamente se puede definir lo que es una \emph{supersolución de viscosidad} de \eqref{parabolic-non-linear}. Una \emph{solución de viscosidad} es una función que es tanto subsolución como supersolución. 

\subsection*{Objetivos : } 
\begin{itemize}
    \item El primer objetivo será probar la recurrencia \eqref{nueva-recurrencia}, la cual permitirá establecer una conexión con la versión discreta \eqref{discreto} de la ecuación totalmente no lineal que nos interesa resolver. Se realizarán experimentos numéricos para verificar los resultados teóricos e identificar aplicaciones significativas de este nuevo método.
    \item El segundo objetivo será adaptar la teoría de Crandall, Ishii y Lions \cite{crandall1992users} para ecuaciones parábolicas totalmente no lineales como \eqref{parabolic-non-linear}. Este análisis se cubre brevemente en \cite{crandall1992users} y \cite{peng2010nonlinear}, aunque hay muchos detalles que no se discuten. Dado que la no linealidad de la ecuación \eqref{parabolic-non-linear} sólo depende del laplaciano, es de esperar que el análisis se simplifique significativamente. 
    \item El tercer objetivo será aproximar la solución de viscosidad de \eqref{parabolic-non-linear} con soluciones de la ecuación discreta \eqref{discreto}. Esto se realizará aproximando el movimiento browniano por medio de caminatas aleatorias, utilizando argumentos similares a los del teorema de Donsker.
    \item El cuarto objetivo será extender los resultados de los incisos anteriores para el caso del laplaciano fraccionario. 
\end{itemize}

\section{Resultados esperados}

La recurrencia \eqref{nueva-recurrencia} es equivalente a la relación de Wald-Bellman tomando como condición inicial a $V^0 = G$. Sin embargo, \eqref{nueva-recurrencia} permite establecer una conexión profunda entre problemas de paro óptimo y ecuacions parabólicas totalmente no lineales. Las soluciones de viscosidad presentan diversas ventajas, en particualr, que son robustas ante aproximaciones. Esto se debe a que funciones meramente continuas se pueden interpretar como soluciones a ecuaciones totalmente no lineales, haciendo posible aproximarlas sin necesidad de controlar la convergencia de sus derivadas. Se espera que este análisis se simplifique en comparación al caso en el que la no linealidad de \eqref{parabolic-non-linear} depende de todos los datos del problema, como ocurre en \cite{katzourakis2014introduction}. No obstante, lo novedoso de este proyecto es establecer una relación con el problema de paro óptimo \eqref{paro-optimo} en el caso continuo.  

Dado que la teoría de Crandall, Ishii y Lions aplicada a nuestro caso sólo requiere que la no linealidad sea monótona con respecto al argumento en $\mathbb{S}(N)$, uno podría cuestionarse qué ocurríria si tomáramos el laplaciano fraccionario $\Delta^\alpha$, $\alpha>0$, en  \eqref{parabolic-non-linear}, en lugar del laplaciano clásico. En esta nueva situación se espera obtener procesos de Lévy no lineales (el laplaciano fraccionario es generador infinitesimal de una gran variedad de procesos de Lévy). Al introducir el laplaciano fraccionario surgen varias preguntas que hasta el momento se encuentran abiertas, siendo una de ellas, ¿qué problemas de paro óptimo están relacionados a las correspondientes EDPs parabólicas fraccionarias no lineales?




\section{Dudas (esto no se incluirá en el documento oficial)}
\begin{itemize}
    \item ¿Cómo se calcula el tiempo de paro óptimo con el enfoque de soluciones de viscosidad?
    \item ¿El laplaciano fraccionario satisface las hipótesis de \cite{crandall1992users}?
\end{itemize}





\bibliographystyle{alpha}
 \bibliography{refs}

\end{document}
